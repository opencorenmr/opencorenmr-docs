%\documentclass{elsart}
% Use the option doublespacing or reviewcopy to obtain double line spacing
\documentclass[doublespacing]{elsart}
%\documentclass[preprint]{elsart}

\usepackage{graphicx}

% The amssymb package provides various useful mathematical symbols
\usepackage{amssymb}
\usepackage{amsmath}
%-------------------------------------------------------------------------------
%-------------------------------------------------------------------------------

\begin{document}

%-------------------------------------------------------------------------------

\begin{frontmatter}
%
\title{Untitled yet}
%
\author[kyoto]{Kazuyuki Takeda\corauthref{cor}},
\corauth[cor]{corresponding author}
\ead{takezo@kuchem.kyoto-u.ac.jp}
%\author[kyoto]{Shintaro Satake},
%\author[kyoto]{Yasuto Noda},
%\author[kyoto]{K. Takegoshi},
\address[kyoto]{Division of Chemistry, Graduate School of Science, Kyoto University, 606-8502 Kyoto, Japan}
\begin{abstract}
Not written yet.
\end{abstract}

\begin{keyword}
% keywords here, in the form: keyword \sep keyword
Raney cobalt; Colabt-59; ferromagnetic resonance; internal field;
\end{keyword}
\end{frontmatter}


%------------------------------------------------------------------------------
\clearpage

\section{Introduction}

日本語


\section*{Acknowledgements}
This work was supported by ...


%\begin{thebibliography}{99}



%\end{thebibliography}

%------------------------------------------------------------------------------
%------------------------------------------------------------------------------
%----------------------------       Tables       ------------------------------
%------------------------------------------------------------------------------
%------------------------------------------------------------------------------




%------------------------------------------------------------------------------
%------------------------------------------------------------------------------
%----------------------------  Figure Captions   ------------------------------
%------------------------------------------------------------------------------
%------------------------------------------------------------------------------
%\newpage
%\noindent
%{\large Figure Captions}\\


%\noindent
%{\bf Figure 1}\\
%(a) The receptivity versus the resonance frequency in a magnetic field of 7 T for several nuclear spin species. The receptivity was calculated according to $x_k |\gamma_k|^3 I_k (I_k + 1)$ and normalized by the proton receptivity. (b) The Constant-Frequency Receptivity (CFR) plotted as a function of the resonance field at a constant frequency of 30 MHz.


%------------------------------------------------------------------------------
%------------------------          Figure 1     -------------------------------
%------------------------------------------------------------------------------
%\newpage
%\begin{figure}
%\begin{center}
%\includegraphics[scale=0.8]{fig-sequence.eps}
%\label{fig:sequence}
%\end{center}
%\end{figure}
%
%\vspace*{5mm}
%\begin{minipage}{8cm} ~ \end{minipage}
%\begin{minipage}{8cm}
%{\bf Fig. \ref{fig:sequence}}\\
%\footnotesize{Elemental analysis...\\ Takeda et al.}
%\end{minipage}

%
%
%------------------------------------------------------------------------------
%------------------------          Figure 1     -------------------------------
%------------------------------------------------------------------------------
%\newpage
%\begin{figure}
%\begin{center}
%\includegraphics[scale=0.6]{fig1.eps}
%\caption{}
%\label{fig:receptivity}
%\end{center}
%\end{figure}
%
%\vspace*{5mm}
%\begin{minipage}{8cm} ~ \end{minipage}
%\begin{minipage}{8cm}
%{\bf Fig. 1}\\
%\footnotesize{Untitled...\\ Takeda et al.}
%\end{minipage}
%
%
%
%
%------------------------------------------------------------------------------
\end{document}
%------------------------------------------------------------------------------
